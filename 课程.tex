\documentclass{article}
\usepackage[UTF8]{ctex}
\usepackage{geometry}
\usepackage{natbib}
\geometry{left=3.18cm,right=3.18cm,top=2.54cm,bottom=2.54cm}
\usepackage{graphicx}
\pagestyle{plain}	
\usepackage{setspace}
\usepackage{caption2}
\usepackage{datetime} %日期
\renewcommand{\today}{\number\year 年 \number\month 月 \number\day 日}
\renewcommand{\captionlabelfont}{\small}
\renewcommand{\captionfont}{\small}
\begin{document}

\begin{figure}
    \centering
    \includegraphics[width=8cm]{upc}

    \label{figupc}
\end{figure}

	\begin{center}
		\quad \\
		\quad \\
		\heiti \fontsize{45}{17} \quad \quad \quad 
		\vskip 1.5cm
		\heiti \zihao{2} 《计算科学导论》课程总结报告
	\end{center}
	\vskip 2.0cm
		
	\begin{quotation}
% 	\begin{center}
		\doublespacing
		
        \zihao{4}\par\setlength\parindent{7em}
		\quad 

		学生姓名:\underline{\qquad  陈鑫雨 \qquad \qquad}

		学\hspace{0.61cm} 号:\underline{\qquad 1907010301\qquad}
		
		专业班级:\underline{\qquad 计科1903 \qquad  }
		
        学\hspace{0.61cm} 院:\underline{计算机科学与技术学院}
% 	\end{center}
		\vskip 2cm
		\centering
		\begin{table}[h]
            \centering 
            \zihao{4}
            \begin{tabular}{|c|c|c|c|c|c|c|}
            % 这里的rl 与表格对应可以看到,姓名是r,右对齐的;学号是l,左对齐的;若想居中,使用c关键字。
                \hline
                课程认识 & 问题思 考 & 格式规范  & IT工具  & Latex附加  & 总分 & 评阅教师 \\
                30\% & 30\% & 20\% & 20\% & 10\% &  &  \\
                \hline
                 & & & & & &\\
                & & & & & &\\
                \hline
            \end{tabular}
        \end{table}
		\vskip 2cm
		\today
	\end{quotation}

\thispagestyle{empty}
\newpage
\setcounter{page}{1}
% 在这之前是封面,在这之后是正文
\section{引言}
《计算机科学导论》就科学特点,学科形态,历史渊源,发展变化,典型方法,学科知识组织结构和分类体系,各年级课程的重点,以及如何认识计算机科学,学好计算机科学等问题从科学哲学和高级科普的角度去回答大家的疑问,因而难以起到后续课程引导的作用。今天,由于经济的快速发展和外来生活方式,文化的影响,大学的浮躁的情绪不免有所抬头,计算机科学与技术专业教育在不少学校出现“文科化”的现象经有几家计算机产品制造商制定了向世界经济五百强企业进军的计划,但它们能否培养出融中西文化于一体,掌握现代高级计算机科学与技术的世界一流人才。因为,若干年之后,我国计算机科学与技术包括产业能否赶上国外先进水平,最终要靠今天和未来培养的一代又一代新人去实现。他们能否肩负重任,取决于大学本科学习是基础是否厚实。而教学计划与课程体系是否科学,最终要由实践来检验。\par

\section{对计算科学导论这门课程的认识、体会}
 整体认识:这门课程就是教会我们认识计算机这个行业的前景,提升我们对计算机行业的兴趣,让我们对自己的专业有更深入的认识,并且掌握科学的方法论,让我们在生活工作中形成自己的方法论。同时老师鼓励我们要做自己不敢做的事,不愿意做的事和做你没做过的事。但是这堂课的理论知识很强,很多知识只是提及,我们想要了解还需要自己课下的研究,但却介绍了很多不同关于计算机行业可以发展的方向,给我们未来的发展方向有启迪作用。\par
例子(1):老师让同学们选择自己感兴趣的话题,两个人合作共同研究,并在同学们面前展示自己的研究成果。(让研究的同学更加了解自己感兴趣的方向,并且让同学们也了解了计算机的不同领域,拓展了同学们的视野,了解计算机的前景与方向,同时锻炼了同学们上台展示自己的能力和制作PPT的能力。)\par
例子(2):老师会在课上讲解关于计算机的各种方面的知识,拓展我们的视野。教我们了解自己的专业培养方向,有自己的职业规划。\par
自己的阅读:\par
例子(3):让我认识到程序员年纪变大之后会面临的问题,并引发我对自己未来职业方向的思考。(其中,我想当一名大学老师;或者是一名程序员,但是年龄大之后可能会考虑转到管理层,形成自己的管理方案)\par
处于对VR成像的好奇心我阅读了《门镜结构及光学原理》,后来我处于对VR佩戴舒适度的好奇,阅读了《基于OCULUS VR全景立体视频的视觉设计舒适度探究》了解了其中的一些原理。后来我对于如何提高主管视觉质量的提高《基于无空洞填补的DIBG方法》。\par

\section{进一步的思考}
结合学习的计算科学知识,对分组演讲涉及的问题作进一步的思考。\par
VR眼睛佩戴过程中产生的眩晕感如何解决?\par
产生原因:眩晕是用户在使用头部控制虚拟现实系统后产生的,其中主要由视觉系统引起的眩晕感,因头显本身的刷新率、闪烁、陀螺仪等引起的高延迟问题导致。\par
解决办法:\par
1.低延迟技术\par
很重要的一个指标是从转动头部到转动画面的延迟。研究表明,头动和视野的延迟不能超过20ms,不然就会出现眩晕。\par
2. 添加虚拟参考物\par
普杜大学计算机图形技术学院的研究人员发现,只要在 VR 场景中加一个虚拟的鼻子,就能解决头晕等问题。(不会形象自己的体验,就行我们现实生活中也有鼻子,但是自己不会察觉到)\par
3.电前庭刺激\par
将电极放在策略性位置(每只耳朵后要放置两个电极,一个在前部,一个在颈背),追踪用户内耳的感知运动,并将视野范围的运动触发成GVS同步指令,刺激产生三维运动。如果行得通的话,它可以让用户完全沉浸在当前的环境中,真正感觉在自己驾驶的宇宙飞船在俯冲或转弯。\par
4. 调节镜片之间的距离\par
利用了滑轮调节镜片之间距离的设计,可以自由调节两个镜片之间的距离。另外,也有方案显示可以通过蓝牙控制器等调节画面的中心点。从而保证画面中心、镜片中心、人眼中心三点一线。避免重影,避免晕眩。\par
5. 光场摄影\par
直接把整个数字光场投射到使用者的视网膜上,从而可以让使用者可以根据人眼的聚焦习惯自由地选择聚焦的位置,以准确的虚实结合模拟人眼的视觉效果。\par
https://blog.csdn.net/tcpipstack/article/details/52024537\par
6. 反畸变矫正 
在沉浸式环境里,画面快速变化,如果看到图像与现实中 的畸变较大,也是大脑产生眩晕的原因之一。图像产生畸变的原因是不同凸透镜的球面位置的焦距不同, 所示远离中 轴线的光线焦急越短,根据凸透镜的成像原理,焦距 f 越短放大 倍数越大,所以远离中心点的图像看上去就越被拉伸,所以通 过凸透镜放大后的物体,人眼看到的虚像所示会产生一 个枕形畸变的效果。\par
为了让人眼看到正常的图像,我们对图像进行反畸变处 理,就是预先把图像进行桶形畸变处理,根据畸变方程 S= 1+ K1*r2 +K2*r4,其中 K1,K2 表示透镜的畸变参数,根据不同透镜 设定,r 表示屏幕各像素距离屏幕中心点的距离。\par
7. 异步时间扭曲 (Asynchronous Timewarp 简称 ATW)技术 \par
ATW 是一种生成中间帧的技术,当应用不能保持足够帧率 的时候,ATW 会根据当前最新的头部位置,对上一帧进行扭曲 处理能产生中间帧,从而有效减少画面的抖动。该技术是降低 延迟,解决眩晕的最为关键的技术。 如果在每次垂直同步前没有新的渲染帧输 出,那么就会导致前一帧 FrameN-1 数据被刷新 2 次,当用户从 30°的位置旋转到 31°时,屏幕上仍然显示 30°的图像,那么大脑 反应是转到 31°,但眼睛受接收到是 30°的内容,这样二者的信 息不匹配,就晕了。\par
异步时间扭曲在每一次垂直同步前,异步进 程根据渲染进程最新完成的那一帧生成一个新的帧如图红色 标注,保证在每一次垂直同步前都会根据最新头部位置产生新帧,这样做能够获取得更低的延迟以及更高的准确度。\par
8. 单 buffer 渲染技术(Front Buffer Render) 
Android 显示框架采用双 buffer 循环机制,一块 buffer 在绘 制(Producer),另外一块 buffer 提交显示(Consumer),只有当 Producer 完全准备好,Consumer 才能使用。对于 android 的显示 框架,APP 显示还要经过 urfaceflinger 
提交到显示屏,android 的绘制到显示流程,每一个步骤都按 vsync 的节拍进行,这样从 绘制到显示出来要经过 3 个vsync。显示屏假定按 60HZ 刷新, 也就是一帧的刷新周期为 16.7ms,所以原生的显示延迟接近 50ms。采用 FBR(Front Buffer Render)没有双 buffer 循环,绕过 surfaceflinger 直接到显示屏,可以大幅减少显示的延迟。\par
怎样创造和寻找VR的赢利点?\par
1.交互式体验,营造身临其境之感\par
随着AI、VR等技术的快速发展,图片、视频和图文等单向传递的方式已经无法满足客户对企业,产品真实信息的需求。人们更迫切地想知道这家企业是否真实存在,是否有能力提供相应的服务或产品,以及真实样式如何等信息。利用道可云VR全景可以实现1:1的真实化、逼真化再现,从厂区入口到产品介绍,再到下单购买,使用户全程参与其中。VR营销的沉浸式体验和深度交可以给用户带来身临其境之感,带来真实的交互体验。\par
2.产品的3D立体化展现\par
VR全景在展示应用中显示效果可以做到上、下、左、右、前、后720°的全部场景、无死角的立体化展示(可进行放大、缩小)。其中还可以加入产品的电子画册、产品解说、对接第三方平台(极大的节约了传统企业的营销宣传成本)等功能,推动平面广告向沉浸式广告的转变,让品牌拥有灵魂。如海信集团VR全景(道可云制作)。\par
3.深度互动,促进消费\par
VR全景还提供了全程语音解说,在线互动说说(类似于淘宝商品的评价),可以快速拉近企业与消费者之间的距离,降低沟通效果。VR的沉浸性和互动性,将品牌融入消费者的生活场景中,建立起品牌与消费者的情感纽带,引起消费者的共鸣和需求。如京东、淘宝的VR购物!\par
4.多渠道宣传,多屏互动扩展\par
VR全景营销实际上是以VR技术为核心的整合营销,它通过VR技术、全景内容、硬件设施、宣传渠道等,为企业打造整合营销方案。道可云,VR全景可与QQ、微信、Facebook、领先等国内外主流社交平台实现无缝对接,支持二维码分享,并可自适应设备。其强大的嵌入、分享功能,打通了设备到传播的通道,实现全媒体的多屏互动,实现消费者与产品的无缝对接。\par
VR全景720°的全视角展示,使其更接近真实,产品更形象,更具宣传推广价值,更好地将品牌的故事性与情感传递给消费者,实现品牌与用户之间的深度交互。\par
5. 广告收入\par
VR影视具有沉浸式、交互性、构想性的特点, 这就要求VR影视在制作时要兼顾内容和技术的融合。鉴于目前VR技术还不成熟, VR影视作品造价较高, 所以在制作VR影视作品时为了保证影片的高质量完成, 除影视公司本身的投资外, 还需要在前期吸引投资。这方面吸引的投资主要来自于广告投资、众筹、风险投资等, 其中, 重点是依靠广告的投资来获取收益。VR影视的广告投资形式与传统影视一样, 也可以设置映前广告、贴片广告或是植入广告。\par
6.线上线下平台运营收入\par
由于VR影视使用的是虚拟现实这种高科技技术, 所以通过线上渠道做平台运营并进行应用分发, 使用户直接从线上渠道购买VR影视作品对VR影视来说是最好的盈利模式。这种盈利模式的优点在于其宣发成本极低, 并且借助发售平台的口碑名气可以为作品本身制造流量。这种盈利模式主要也分为三种类型:一种是所售即所得, 即由于线上平台想要赚取流量等理由不向VR影视CP方收取费用, 也不与CP方分利;第二种是按协议盈利, 这种类型是最常见的, CP方与平台签订协议, 平台在前期收取宣发费用或在后期根据发售金额按一定比例抽取佣金等等;第三种是一次性支付, CP方将VR影视作品售予平台收取盈利并完全脱离作品后期的盈利。\par
7.衍生品开发\par
在VR影视衍生产品开发阶段, 就传统影视行业来说, 衍生产品开发阶段盈利模式较为简单, 即根据影片内容开发相关周边产品。在VR影视的衍生品开发中, 这同样是最重要的一点。由于观众在VR影视体验中可以更近距离接触剧情、故事人物以及影片其他相关设置, 这会使得观众对由VR影视作品开发出的衍生产品更加熟悉和亲切, 因此VR影视衍生产品的售卖也是VR影视获取收益的重要途径之一。由于传统国产电影对于衍生产品的开发并不充足, 因此, 在VR影视的衍生产品开发上, 可以借鉴好莱坞的衍生产品开发模式, 即在影片制作前讨论发现和设置衍生品内容并联系好相关厂商, 以便在影片上映后迅速推出影片相关衍生产品以获取收入。这样不仅可以延长扩散影片的热度, 增加影片的票房盈利, 还能通过衍生产品的售卖获取相关盈利。\par
畅想VR的未来:\par
(1)对电影电视的变革:\par
 不用守着个电视机了,VR电影电视让你身临其境,一起和主角成长,沉浸感更强,而且不同的视角还能看到不同的东西,个人觉得推理志什么的最适合用VR的方式来拍了。而且VR PRON在国外非常火爆,这简直就是所有宅男的福音。\par
(2)对游戏的变革:\par
就在几年前,《网络之XXX》这类小说大行其道,里面所描述的游戏沉浸感十足,主角在网络游戏里面走向人生巅峰,而VR技术普及之后,就完全可以实现以前的YY。\par
还有对工业,医疗,教育,各行各业的变革都将会是彻底的,VR交互将会渗透进社会的每一个角落。 到那个时候,空间真的就不再是问题了,人们在世界每一个角落都可以紧密无间。\par
(3)VR/AR只是对交互方式的变革,而不是对所有技术的颠覆。而且它只是以后万物互联的一部分 VR/AR都不会一枝独秀,而会并行发展,并结合使用,就像上述提到的场景,旅游,在未来,导游自己带上AR眼镜,把实时渲染的数据共享给VR用户进行渲染,就可以网上旅游了,这种场景就需要AR/VR的配合,AR实时渲染,VR完全沉浸.所谓MR就是这两者的实现了吧。\par
原文链接:https://blog.csdn.net/wolfqong/article/details/51984377\par



\section{总结}
在这里,写自己对于整个课程和或本次报告的总结。\par
这堂课交给我不仅要学习专业知识,还要有一种前瞻意识,要了解自己所学专业的前景与方向,教会了我团队意识和做ppt的能力,也让我了解了计算机很多有趣有前景的行业,同时我们要形成一套自己做事的方法论,要做自己不敢做的事,不愿意做的事,没做过的事。同时了解自己感兴趣的方向自己所能利用的资源,能请教的老师,能利用的网站,值得学习的文件,自己考虑问题的方法。这堂课重在讲理论知识,里面很多知识只是提及了一下,如果要深入了解还需要我们自己课下来研究,这本书,我们也应该课下来多翻一翻,当故事书来看,重在一个启迪作用。\par


\section{附录}
\begin{itemize}
    \item 申请Github账户,给出个人网址和个人网站截图\par
    Github网址:https://github.com/chenxinyu9/science.git\par
    https://github.com/chenxinyu9/chen\par

\begin{figure}[tph]
	\centering
	\includegraphics[scale=0.4]{Github}
	\caption{Github}
	\label{fig:Github}
\end{figure}
\begin{figure}[tph]
	\centering
	\includegraphics[scale=0.4]{gcz}
	\caption{gcz}
	\label{fig:gcz}
\end{figure}
\begin{figure}[tph]
	\centering
	\includegraphics[scale=0.4]{xuexi}
	\caption{xuexi}
	\label{fig:xuexi}
\end{figure}
\begin{figure}[tph]
	\centering
	\includegraphics[scale=0.4]{bilibili}
	\caption{bilibili}
	\label{fig:bilibili}
\end{figure}
\begin{figure}[tph]
	\centering
	\includegraphics[scale=0.4]{CSDN}
	\caption{CSDN}
	\label{fig:CSDN}
\end{figure}
    
   CSDN网址: https://blog.csdn.net/chenxinyu9
\begin{figure}[tph]
	\centering
	\includegraphics[scale=0.4]{bky}
	\caption{bky}
	\label{fig:bky}
\end{figure}
\begin{figure}[tph]
	\centering
	\includegraphics[scale=0.4]{xmc}
	\caption{xmc}
	\label{fig:xmc}
\end{figure}
\end{itemize}


\hspace*{\fill} \\

参考文献:\par
[1]张晓芸.浅析虚拟现实技术在科普教育中的作用[J].中国科教创新导刊, 2013, (29) :1-1.\par
[2]姚维维.基于OCULUS VR全景立体视频的视觉设计舒适度探究[D].北京:北京交通大学, 2016:33-34\par
[3]高华, 刘君宇, 李萌, 熊蛟, 许玲玲.双凸透镜系统成像规律分析[J].物理与工程, 2016, 26 (5) :41-42.\par
[4]蒋从元.门镜结构及光学原理[J].四川职业技术学院学报, 2008, 18 (3) :122.\par
[5]William R.Mark, Leonard Mc Millan, Gary Bishop.Post-Rendering 3D Warping[J].Symposium on Interactive 3D Graphics, 1997, 7-16:4-\par
[6]姚杰, 陈一民.基于无空洞填补的DIBG方法[J].计算机应用与软件, 2015, 32 (4) :223.\par
[7]Ho Chunling, Dzeng Renjye.Constrction Safety via Learning;Learning Effectivenessand User Satisfaction[J].Computers,Eduati on, 2010 (55) :858-867.\par

\end{document}
